\section{Motivation und Aufgabenstellung}
\label{l:motiv}
Die Softwareentwicklung f�r Mikrocontroller l�sst sich in zwei Phasen
unterteilen. In der ersten Phase, bis zum Abschluss der Implementierung, wird 
auf einem Fremdsystem entwickelt. Durch ihre weite 
Verbreitung kommen zumeist Personalcomputer zum Einsatz. 
Mit Verwendung von Crosscompilern l�sst sich Programmcode f�r den
gew�hlten Mikrocontroller erzeugen. Erst mit der �ber\-tragung des 
Programmcode 
auf den Mikrocontroller beginnt die zweite Phase der Softwareentwicklung, nun
auf dem beabsichtigtem Zielsystem.
\newline
Durch die F�higkeit von Mikrocontrollern zum sog. In-System-Programming
wird der �bergang zwischen beiden Entwicklungsphasen vereinfacht. Auf 
aufwendige 
Umbauma�nahmen der Hardware oder gar einem Ausbau des Mikrocontrollers aus 
einem kompletten System kann verzichtet werden. Vielmehr kann ein
Mikrocontroller, wie der Name schon vermuten l�sst, beim �bertragen von 
Programmcode im System verbleiben. Dies reduziert die mechanische 
Beanspruchung des gesamten Systems, zudem wird  
viel Zeit eingespart. Die gewonnene Flexibilit�t l�sst sich beispielsweise
f�r zus�tzliche Tests im Rahmen der Softwareentwicklung einsetzen.  
Das In-System-Programming ( im Folgendem auch kurz mit ISP bezeichnet) nutzt 
vorhandene Kommunikationsm�glichkeiten eines Mikrocontrollers. Bekannt sind 
serielle M�glichkeiten wie JTAG oder SPI. Ein entsprechendes Kabel samt
Adapter stellt dabei die Verbindung zwischen PC und 
Mikrocontroller her. 
\newline
Mikrocontroller der AT90CAN-Baureihe der Firma Atmel 
besitzen eine Schnittstelle f�r den CAN-Bus. Wird diese Schnittstelle als 
vornehmliche Kommunikationsm�glichkeit verwendet, erscheint die Verwendung 
einer zus�tzlichen Schnittstelle lediglich zum Zwecke des ISP �berfl�ssig.
\newline
Die M�glichkeit zum ISP �ber den existierenden CAN-Busanschluss bietet mehrere
Vorteile:
\begin{itemize}
        \item Es entf�llt das Anschlie�en eines 
        zus�tzlichen Verbindungskabels, wenn aus\-schlie�\-lich per CAN 
        kommuniziert wird. Zus�tzlich bleibt in diesem Falle eine 
        Schnittstelle des Mikrocontrollers st�ndig f�r andere Aufgaben frei.
        \item Es entf�llt der oft l�stige Wechsel des jeweils angeschlossenen 
        Mikrocontrollers. Bisherige Verfahren erlauben nur die Kommunikation 
        mit einem einzelnen Mikrocontroller. 
        F�r die Verwendung mehrerer Mikrocontroller muss bisher immer das 
        Verbindungskabel getrennt und neu angeschlossen werden. 
        \item Es wird zus�tzlich Zeit eingespart, da mehrere Mikrocontroller
        gleichzeitig mit neuem Programmcode versehen werden k�nnen.
        Gerade bei Verwendung mehrerer Mikrocontroller nimmt das bisher 
        praktizierte sequentielle Vorgehen viel Zeit in Anspruch.
        \item Letztendlich wird auch die mechanische Beanspruchung der 
        Anschl�sse reduziert. 
\end{itemize}
Ziel des durchgef�hrten Laborpraktikums war es, die genannten Vorteile durch
eine entsprechende Entwicklung nutzbar zu machen. Grundlage waren, wie bereits 
angedeutet, Mikrocontroller des Typs Atmel AT90CAN128. 
Die Aufgabe umfasste Konzeption eines Befehlprotokolls und 
entsprechende Programmierung der Mikrocontroller. Weiterhin galt es, einen 
entsprechenden Gegenpart auf Seite eines PC zu entwickeln. Als Resultat 
sollte der eingangs beschriebene Prozess der Softwareentwicklung f�r 
Mikrocontroller weiterhin durchf�hrbar sein.




